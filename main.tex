\documentclass{article}
\usepackage{graphicx} % Required for inserting images

\title{Kermit-wave phenomena}
\author{Loïc Lemieux-Crête}
\date{December 2025}

\begin{document}

\maketitle

\section{Modeling of Wave Phenomena}




Wave phenomena play a central role in both physics and engineering, as they 
provide a fundamental mechanism for energy transfer without net transport of 
matter. Examples include acoustic propagation in air, stress waves in elastic 
solids, and surface oscillations in fluids. In the context of simulation, waves 
are also critical for modeling vibrational effects in robotics, resonances in 
mechanical systems, and visually plausible deformations in computer graphics.

In discretized systems composed of particles or dynamic units, wave-like 
behavior can be represented in multiple ways. Traditionally, partial 
differential equation (PDE) discretizations are employed, where local 
interactions approximate continuous differential operators~\cite{macklin2014, 
muller2003}. In contrast, this work also introduces a novel analytical 
formulation in dual quaternion space, designed for direct compatibility with 
rigid-body and pose-based representations~\cite{kavan2007, zhu2005}.


\subsection{Scalar and Discrete Formulation}

In a homogeneous isotropic medium, a linear harmonic wave is governed by the 
classical scalar wave equation~\cite{courant1962}:
\begin{equation}
\frac{\partial^2 u}{\partial t^2} = c^2 \nabla^2 u,
\end{equation}
where $u(\mathbf{X},t)$ is the displacement field, $c$ the wave velocity, and 
$\nabla^2$ the Laplacian operator.

In particle-based discretizations, the Laplacian is typically approximated 
through local neighborhood differences~\cite{gingold1977, liu2003}:
\begin{equation}
\nabla^2 u_i \approx \frac{1}{h^2} 
\sum_{j \in \mathcal{N}(i)} (u_j - u_i),
\end{equation}
where $\mathcal{N}(i)$ denotes the neighborhood of particle $i$, and $h$ is the 
characteristic spacing.

A localized impulse may be expressed as a Gaussian-modulated sinusoid 
\cite{morse1968}:
\begin{equation}
u_0(\mathbf{X}) = 
A \exp\!\left(-\frac{\|\mathbf{X}-\mathbf{X}_0\|^2}{\sigma^2}\right)
\sin\!\left(k\|\mathbf{X}-\mathbf{X}_0\| + \phi\right).
\end{equation}

Explicit numerical integration must satisfy the 
Courant--Friedrichs--Lewy (CFL) condition~\cite{courant1928}:
\begin{equation}
\Delta t < C \frac{h}{c},
\end{equation}
with $C$ depending on the discretization scheme.

While this PDE-based approach is physically faithful, it requires careful 
spatial resolution and becomes computationally expensive for high-frequency or 
high-dimensional simulations.


\subsection{Dual Quaternion Wave Field (Proposed)}

To extend wave modeling beyond scalar displacement fields, we define a wave 
directly in dual quaternion space, enabling consistent perturbations of both 
translations and rotations in $\mathrm{SE}(3)$. Let 
$\hat{\xi} \in \mathbf{SE}(3)$ denote a unit dual twist (screw axis) 
specifying the local displacement/rotation direction. The resulting wave field is:
\begin{equation}
Q(\mathbf{X},t) = 
Q_0(\mathbf{X}) \otimes 
\exp\!\big( \beta(\mathbf{X},t)\, \hat{\xi} \big),
\end{equation}
where $Q_0(\mathbf{X})$ is the undeformed pose field and $\exp$ the dual 
quaternion exponential.

The scalar modulation factor $\beta(\mathbf{X},t)$ encodes the physical wave parameters:
\begin{equation}
\beta(\mathbf{X},t) =
\frac{A}{\rho}
e^{- v t}
\sin\!\left( k (\hat{n} \cdot \mathbf{X}) - \omega t + \phi \right).
\end{equation}

Here:
\begin{itemize}
    \item $A$ is the wave amplitude,
    \item $\rho$ is the local density,
    \item $v$ is viscosity/damping,
    \item $k = 2\pi/\lambda$ is the wave number,
    \item $\omega = 2\pi f$ is the angular frequency,
    \item $\phi$ is the initial phase,
    \item $\hat n$ is the propagation direction,
    \item $\hat \xi$ is the screw axis coupling rotation and translation.
\end{itemize}

This construction captures the physical behavior of the medium while lifting 
the oscillation into $\mathrm{SE}(3)$ via the exponential map.


\subsubsection{Eulerian Incremental Alternative}

For world-space propagation, an incremental update is:
\begin{equation}
Q_{t+\Delta t} = 
Q_t \otimes 
\exp\!\big( \Delta t\, \gamma(Q_t,t) \hat{\xi} \big),
\end{equation}
with $\gamma$ evaluated at the current pose.

Stability requires a CFL-like bound ensuring that wave propagation speed does 
not exceed the spatial discretization scale.


\subsection{Discussion}

This dual quaternion wave field differs fundamentally from Laplacian-based PDE 
methods. Instead of approximating differential operators through neighbor 
interactions, the field is expressed analytically and mapped directly onto the 
Lie group $\mathrm{SE}(3)$.

Advantages include:
\begin{itemize}
    \item consistent translation/rotation coupling via $\hat{\xi}$,
    \item incorporation of physical material factors in $\beta$,
    \item compatibility with both Lagrangian and Eulerian interpretations,
    \item lightweight evaluation without PDE solvers,
    \item geometric coherence for robotics and animation tasks.
\end{itemize}


\subsection{Related Work}

Wave phenomena are traditionally modeled using PDEs in Euclidean space, such as 
the scalar and vector wave equations. Numerical approaches include finite 
difference (FDTD), finite element (FEM), and particle-based schemes including 
SPH and MPM. These methods approximate energy transport through local 
interactions but come with significant computational and stability demands.

Dual quaternions have become a standard tool for representing rigid-body motion 
in robotics and computer graphics~\cite{kavan2007}. While Lie-group integrators 
have been applied to dynamics on $\mathrm{SO}(3)$ and $\mathrm{SE}(3)$, prior 
work does not treat wave propagation in dual quaternion space. Procedural 
oscillation methods exist in graphics, but they operate in $\mathbf{R}^3$ 
without leveraging screw transformations or embedding physical parameters such 
as density or damping.

To our knowledge, no previous formulation defines a wave field directly in dual 
quaternion space. The proposed method offers a geometrically coherent, 
physically meaningful, and computationally lightweight alternative to PDE-based 
wave solvers for robotics and animation.

%Sets the bibliography style to UNSRT and imports the 
%bibliography file "sample.bib".

\bibliographystyle{unsrt}
\bibliography{sample}

\end{document}

